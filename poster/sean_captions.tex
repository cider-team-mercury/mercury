\documentclass[a4paper,10pt]{article}
\usepackage[utf8]{inputenc}
\usepackage{amsmath}

%opening
\title{Poster Captions}
\author{}
\date{}

\begin{document}

\maketitle


% profiles.png
We have developed a 3-layer interior structure model for mercury, with S in the core.
Material properties are calculated using a Mie-Gr\"{u}neisen EOS, using the BURNMAN
lower mantle mineral physics tool kit. It determines adiabatic temperature profiles
consisten with the pressure of the inner core boundary and the composition of the
liquid. Shown above are iterative solutions to the pressure, density and gravity
profiles, which allow us to determine energy balance of the planet and compare models
with physical parameters for present-day mercury.

% Liquidus_model.png
A model liquidus is necessary for relating the temperature profile and light element
flux in the core of the planet. Shown above is our model liquidus, fit to
experimentally determined phase diagrams. The melting curve of the Fe-S system has
enigmatic features, that may not be captured captured by a simple linear
interpolation with composition between the Fe-melting curve and eutectic. The onset
of ``snow'' regions in the core is extremely sensitive to this interpolation.

% Clapeyron_1.png
Shown above is the slope of the melting temperarature for pure iron, alongside that
of our interpolated model. Comparing this the slope of the adiabatic profile
determines the crystallization behavior of the core, with a steeper melting
temperature corresponding to  ``Earth-like'' crystallization. We find that for higher
values of the coeffecient of thermal expansivity $\alpha$ foud in the literature
, the core will be ``snowing'' at all times. For lower values
onset of snowing occurs with increasing S-content, but is extremely sensitive to the
liquidus model used.

% core_energetics.png
The interior structure model can be coupled to a parameterized convetion model for
the thermal evolution of the planet. Shown above are the changes in gravitational
and thermal energy in the core, along with the latent heat relase by core growth per
change in temperature of the core mantle boundary for a core with a bulk composition
of 6 wt.\% S. For a given composition $T_{cmb}$ is directly related to the size of
the core. Shown above are the maximum and best-fit inner core radius from
Dumberry \& Rivoldini 2015.



\end{document}
