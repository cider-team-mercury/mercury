\documentclass[a4paper,10pt]{article}
\usepackage[utf8]{inputenc}
\usepackage{amsmath}

%opening
\title{Heat flux maps}
\author{}
\date{}

\begin{document}

\maketitle

\section{ The simple case}


If the mercurian mantle is not convecting, and has not been for a long time, we would expect the mantle temperature distribution to approach a conductive solution.  
However, due to the unusual spin-orbit resonance in which Mercury is locked, there are large and persistent temperature variations on the surface.
This is because some regions get significantly more insolation than others.
\\
\\
Here I calculate the temperature distribution in a conducting mantle with constant conductivity and temperature boundary conditions.  
I assume that the temperature at the CMB is constant due to efficient transport of heat in the liquid outer core.  
Further, I assume that the temperature at the surface is time-independent, but may vary spatially due to insolation differences.  
This assumes that I am taking the surface to be below the skin-depth of the regolith.

\begin{table}[h]
\centering
\begin{tabular}{l l}
Outer radius & $R_o$ \\
Inner radius & $R_i$ \\
Aspect ratio & $R_i/R_o = \eta$ \\
Inner temperature & $T_i$ \\
Surface temperature & $S(\theta, \phi)$ \\
\end{tabular}
\caption{Symbols used}
\end{table}


\subsection{General solution}

Steady state diffusion satisfies Laplace's equation:

\begin{equation}
\nabla^2 T = 0
\end{equation}

In spherical coordinates this has the general solution

\begin{equation}
T(r, \theta, \phi) = \displaystyle \sum_{l=0}^{\infty} \sum_{m=-l}^{l} \left[ A_{lm} r^l + B_{lm} r^{-(l+1)} \right] Y_{lm}(\theta,\phi)
\end{equation}

Where $Y_{lm}$ represents fully normalized real spherical harmonics, and $A_{lm}$ and $B_{lm}$ are coefficients.

We suplement this with the boundary conditions

\begin{equation}
T(R_o, \theta, \phi) = \displaystyle \sum_{l=0}^{\infty} \sum_{m=-l}^{l} \left[ A_{lm} R_o^l + B_{lm} R_o^{-(l+1)} \right] Y_{lm}(\theta,\phi) = S(\theta, \phi)
\end{equation}

\begin{equation}
T(R_i, \theta, \phi) = \displaystyle \sum_{l=0}^{\infty} \sum_{m=-l}^{l} \left[ A_{lm} R_i^l + B_{lm} R_i^{-(l+1)} \right] Y_{lm}(\theta,\phi) = T_i
\end{equation}


\subsection{Solution for $l \ne 0$}

The inner boundary has no $\theta, \phi$ dependence, so the function of $r$ multiplying the spherical harmonics must be zero for $l \ne 0$:

\begin{equation}
\begin{aligned}
&A_{lm} R_i^l + B_{lm} R_i^{-(l+1)} = 0 \\
&A_{lm} R_i^l = - B_{lm} R_i^{-(l+1)} \\
-&A_{lm} R_i^{(2l + 1)} = B_{lm} \\
\end{aligned}
\end{equation}

We can plug this in to the upper boundary condition to find


\begin{equation}
\begin{aligned}
\displaystyle \sum_{l=0}^{\infty} \sum_{m=-l}^{l} A_{lm} \left[R_o^l - R_i^{(2l+1)} R_o^{-(l+1)} \right] Y_{lm}(\theta,\phi) &= S(\theta, \phi) \\
\displaystyle \sum_{l=0}^{\infty} \sum_{m=-l}^{l} A_{lm} R_o^{l} \left[1 - R_i^{(2l+1)} R_o^{-(2l+1)} \right] Y_{lm}(\theta,\phi) &= S(\theta, \phi) \\
\displaystyle \sum_{l=0}^{\infty} \sum_{m=-l}^{l} A_{lm} R_o^{l} \left[1 - \eta^{(2l+1)} \right] Y_{lm}(\theta,\phi) &= S(\theta, \phi) \\
\end{aligned}
\end{equation}

If we further expand $S(\theta,\phi) = \displaystyle \sum_{l=0}^{\infty} \sum_{m=-l}^{l} S_{lm} Y_{lm}(\theta,\phi)$ we find 

\begin{equation}
\begin{aligned}
A_{lm} R_o^{l} \left[1 - \eta^{(2l+1)} \right] &= S_{lm} \\
\end{aligned}
\end{equation}

So that

\begin{equation}
\begin{aligned}
A_{lm} = \frac{S_{lm}}{ R_o^{l} \left[1 - \eta^{(2l+1)} \right]} \\
\end{aligned}
\end{equation}

Finally, we may find $B_{lm}$:

\begin{equation}
\begin{aligned}
B_{lm} &= -A_{lm} R_i^{(2l+1)} = \frac{-S_{lm} R_i^{(2l+1)}}{ R_o^{l} \left[1 - \eta^{(2l+1)} \right]} \\
B_{lm} &= \frac{-S_{lm} \eta^{(2l+1)}}{ R_o^{-(l+1)} \left[1 - \eta^{(2l+1)} \right]} \\
B_{lm} &= \frac{-S_{lm}}{ R_o^{-(l+1)} \left[\eta^{-(2l+1)} - 1 \right]} \\
\end{aligned}
\end{equation}

We may plug these expressions for $A_{lm}$ and $B_{lm}$ into the general solution to find

\begin{equation}
\begin{aligned}
T(r, \theta, \phi) &= \displaystyle \sum_{l=1}^{\infty} \sum_{m=-l}^{l} \left[ \frac{S_{lm}}{1-\eta^{(2l+1)}} \left(\frac{r}{R_o}\right)^l -  \frac{S_{lm}}{\eta^{-(2l+1)}-1} \left(\frac{r}{R_o}\right)^{-(l+1)} \right] Y_{lm}(\theta,\phi) \\
 &= \displaystyle \sum_{l=1}^{\infty} \sum_{m=-l}^{l} S_{lm} \left[ \frac{1}{1-\eta^{(2l+1)}} \left(\frac{r}{R_o}\right)^l +  \frac{1}{1-\eta^{-(2l+1)}} \left(\frac{r}{R_o}\right)^{-(l+1)} \right] Y_{lm}(\theta,\phi) \\
 &= \displaystyle \sum_{l=1}^{\infty} \sum_{m=-l}^{l} S_{lm} \left[ \frac{1}{1-\eta^{(2l+1)}} \left(\frac{r}{R_o}\right)^l -  \frac{\eta^{(2l+1)}}{1-\eta^{(2l+1)}} \left(\frac{r}{R_o}\right)^{-(l+1)} \right] Y_{lm}(\theta,\phi)
\end{aligned}
\end{equation}

We may verify that for $r = R_o$ this becomes $S(\theta, \phi)$ and for $r=R_i$ this is zero.

\subsection{Solution for $l=0$}


Now we must do the case for $l=0$.  Using the equation for the inner boundary condition, we find 

\begin{equation}
\begin{aligned}
& \left[A_{00} + B_{00}/R_i \right] Y_{00} = T_i \\
& B_{00} = \left[T_i/Y_{00} - A_{00} \right] R_i\\
\end{aligned}
\end{equation}

Plugging into the outer boundary:

\begin{equation}
\begin{aligned}
& \left[A_{00} + B_{00}/R_o \right] Y_{00} = S_{00} Y_{00}\\
& \left[A_{00} + T_i\eta/Y_{00} -A_{00}\eta \right] = S_{00}\\
& A_{00} (1-\eta) + T_i\eta/Y_{00} = S_{00}\\
& A_{00} =  \frac{S_{00} - T_i\eta/Y_{00} }{1-\eta} \\
\end{aligned}
\end{equation}

Therefore

\begin{equation}
\begin{aligned}
& B_{00} = \left[T_i/Y_{00} - \frac{ S_{00} Y_{00} - \eta T_i}{(1-\eta) Y_{00} }  \right] R_i\\
& B_{00} = \left[\frac{T_i (1-\eta)}{(1-\eta)} - \frac{ S_{00}Y_{00} - \eta T_i}{(1-\eta) }  \right] \frac{R_i}{Y_{00}}\\
& B_{00} = \left[T_i (1-\eta) - S_{00}Y_{00} + \eta T_i  \right] \frac{R_i}{(1-\eta) Y_{00}}\\
& B_{00} = \frac{(T_i-S_{00}Y_{00}) R_i}{(1-\eta) Y_{00}}\\
\end{aligned}
\end{equation}

We can now plug these in to the general solution for $l=0$:

\begin{equation}
\begin{aligned}
T(r, \theta, \phi) &= \left[ \frac{S_{00}Y_{00} - \eta T_i}{(1-\eta)Y_{00}} + \frac{T_i - S_{00}Y_{00}}{(1-\eta)Y_{00}} \frac{R_i}{r} \right] Y_{00} \\
&= \left[ S_{00}Y_{00} - \eta T_i + (T_i - S_{00}Y_{00}) \frac{R_i}{r} \right] \frac{1}{1-\eta} \\
&= \left[ S_{00} Y_{00}(1-\frac{R_i}{r}) + T_i ( \frac{R_i}{r} - \eta ) \right] \frac{1}{1-\eta} \\
\end{aligned}
\end{equation}

Again, we may verify this by plugging in $R_i$ and $R_o$ to so that it satisfies boundary conditions.
Furthermore, we may note that $S_{00}Y_{00}$ is the average surface temperature $T_o$.

\subsection{Heat flux calculation}

The heat flux through the CMB is calculated with Fourier's law:

\begin{equation}
q = -k \nabla T \cdot {\bf{r}} = -k \partial T / \partial r
\end{equation}

where k is the thermal conductivity.  This derivative is evaluated at $R_i$.  First, we calculate the case for $l=0$:

\begin{equation}
\begin{aligned}
\partial T/ \partial r = \left[ \frac{ T_o R_i }{r^2} - \frac{T_i R_i}{r^2} \right] \frac{1}{1-\eta}
\end{aligned}
\end{equation}

Evaluating this at $R_i$ we find:

\begin{equation}
\begin{aligned}
q_{cmb} = k \frac{(T_i - T_o)}{R_i} \frac{1}{1-\eta}
\end{aligned}
\end{equation}

For the case of $l \ne 0$ we find:

\begin{equation}
\begin{aligned}
\partial T / \partial r = \displaystyle \sum_{l=1}^{\infty} \sum_{m=-l}^{l} \frac{S_{lm}}{R_o} \left[ \frac{l}{1-\eta^{(2l+1)}} \left(\frac{r}{R_o}\right)^{l-1} +  \frac{(l+1) \eta^{(2l+1)}}{1-\eta^{(2l+1)}} \left(\frac{r}{R_o}\right)^{-(l+2)} \right] Y_{lm}(\theta,\phi)
\end{aligned}
\end{equation}

Again we evaluate this at $R_i$:

\begin{equation}
\begin{aligned}
q_{cmb} &= -k  \displaystyle \sum_{l=1}^{\infty} \sum_{m=-l}^{l} \frac{S_{lm}}{R_o} \left[ \frac{l \eta^{(l-1)}}{1-\eta^{(2l+1)}} +  \frac{(l+1) \eta^{(2l+1)} \eta^{-(l+2)}}{1-\eta^{(2l+1)}} \right] Y_{lm}(\theta,\phi) \\
&= -k \displaystyle \sum_{l=1}^{\infty} \sum_{m=-l}^{l} \frac{S_{lm}}{R_o} \left[ \frac{l \eta^{(l-1)}}{1-\eta^{(2l+1)}} +  \frac{(l+1) \eta^{(l-1)} }{1-\eta^{(2l+1)}} \right] Y_{lm}(\theta,\phi) \\
&= -k \displaystyle \sum_{l=1}^{\infty} \sum_{m=-l}^{l} \frac{S_{lm}}{R_o} \left[ \frac{(2l+1) \eta^{(l-1)}}{1-\eta^{(2l+1)}} \right] Y_{lm}(\theta,\phi) \\
\end{aligned}
\end{equation}

Note that the integral of a spherical harmonic over a sphere with $l \ne 0$ is zero, so each of these components of the heat flux to not affect the overall flux, but are merely spatial variations on top of the average flux as calculated with the $l = 0$ term.


\end{document}
